% Copyright 2020 Pascal Philipp

%    Licensed under the Apache License, Version 2.0 (the "License");
%    you may not use this file except in compliance with the License.
%    You may obtain a copy of the License at

%        http://www.apache.org/licenses/LICENSE-2.0

%    Unless required by applicable law or agreed to in writing, software
%    distributed under the License is distributed on an "AS IS" BASIS,
%    WITHOUT WARRANTIES OR CONDITIONS OF ANY KIND, either express or implied.
%    See the License for the specific language governing permissions and
%    limitations under the License.


\documentclass[12pt,a4paper]{report}

\usepackage{amsmath}
\usepackage{amssymb}
\usepackage{amsthm}
\usepackage{graphicx}
\usepackage{enumerate}
\usepackage[table]{xcolor}
\usepackage{verbatim}
\usepackage[top=1.2in, bottom=1.2in, left=1.2in, right=1.2in]{geometry}
%\usepackage[notref]{showkeys}

\usepackage{tikz}
\newcommand{\tikzmark}[2]{
	\tikz[overlay,remember picture, baseline] 
	\node[anchor=base] (#1) {$#2$};}

\usepackage{thmtools}
\renewcommand{\listtheoremname}{List of Applications}
\renewcommand\thmtformatoptarg[1]{: #1}

\usepackage{enotez}
\let\footnote=\endnote
\setenotez{list-name={Hints and Answers}, totoc=chapter}

\theoremstyle{definition}
\newtheorem{theorem}{Theorem}[chapter]
\newtheorem{lemma}[theorem]{Lemma}
\newtheorem{corollary}[theorem]{Corollary}
\newtheorem{definition}[theorem]{Definition}
\newtheorem{example}[theorem]{Example}
\newtheorem{properties}[theorem]{Properties}
\newtheorem{exercise}[theorem]{Exercise}
\newtheorem{exercise*}[theorem]{Exercise*}
\newtheorem{remark}[theorem]{Remark}
\newtheorem{sketch}[theorem]{Sketch}
\newtheorem*{application}{Application}

\newcommand{\figbox}[1]{
\begin{center}
	\fbox{\parbox[b][2.5in][t]{0.5\textwidth}{{\scriptsize{#1}}}}
\end{center}}
\newcommand*\rfrac[2]{{}^{#1}\!/_{#2}}
\newcommand{\wtd}[1]{\widetilde{#1}}
\newcommand{\abs}[1]{ \left| #1 \right| }
\newcommand{\rank}{\text{rank}\:}
\renewcommand{\d}{\mathrm{d}}
\newcommand{\e}{\mathrm{e}}
\newcommand{\df}[2]{\displaystyle{\frac{\partial #1}{\partial #2}}}
\newcommand{\ddf}[2]{\displaystyle{\frac{\partial^2 #1}{\partial #2^2}}}
\newcommand{\dff}[3]{\displaystyle{\frac{\partial^2 #1}{\partial{#2}\partial{#3}}}}
\newcommand{\Df}[2]{\displaystyle{\frac{\d #1}{\d #2}}}
\newcommand{\Ddf}[2]{\displaystyle{\frac{\d^2 #1}{\d #2^2}}}
\newcommand{\itgr}[4]{\int_{#1}^{#2} #3 \: \d #4}
\newcommand{\eval}[3]{\left. #1 \, \right|_{#2}^{#3}}

\setcounter{tocdepth}{2}

\title{A Compact Course on Mathematical Methods}
\author{Pascal Philipp}


\begin{document}

\maketitle

\newgeometry{top=1.5in, bottom=1.0in, left=0.7in, right=0.8in}
\noindent
The notes at hand cover the following topics:

\begin{enumerate}
\item Vectors and Matrices
\item Functions of Several Variables
\item Integration
\item Differential Equations
\end{enumerate}

The main prerequiste that is required for being able to work through the
text is familiarity with functions of a single variable and differentiation.
Chapter 2 is then the natural follow-up to that prerequisite.
Differentiation for functions of several variables and applications
such as finding minima or maxima are covered.
Chapter 3 introduces integration; first for functions of one variable and then
for functions of several variables.
That second part, multivariate integration, is rather brief and more of a
quick taster and introduction of concepts rather than a systematic treatment.
Chapter 4 then does the same for differential equations.
You could call these three chapters 'advanced calculus'.

The material in chapter 1, vectors and matrices, isn't usually taught together
with calculus -- it is covered here because it is the other important basic
mathematical methods topic needed in STEM
(with calculus being the first; actually, there is a third such topic:
probability and statistics, which is not covered, but I'd be open to starting
collaborations to add a compact chapter on it).
Chapter 1 isn't a prerequisite for the other chapters, but it makes some of
the multivariate notation prettier.

These notes are application-oriented and the philosophy is to
cover the basic concepts quickly, go through a good amount of examples and
exercises, and then move on. No need to go down rabbit holes.
That's where the 'compact' in the title comes from.
Despite focus on methods for applied computations and solving problems,
a few more theoretical remarks and exercises can be found in the text.
Elements of abstract mathematical rigour can be found as well -- I tried to
do give a taste of this way of thinking in a friendly way.

\bigskip
This document belongs to the GitHub repository
\begin{center}
\texttt{https://github.com/pasc85/MathematicalMethods}
\end{center}
You can find the license there as well as information on how to contribute.

\vfill
\noindent
Copyright 2020 Pascal Philipp. Subject to Apache License 2.0.

\thispagestyle{empty}
\newpage
\restoregeometry

\newgeometry{top=0.8in, bottom=1.0in, left=1.1in, right=1.1in}
\tableofcontents
\thispagestyle{empty}
\newpage
\listoftheorems[ignoreall,show={application}]

\bigskip
\noindent
\texttt{Customise the format: drop ``Application: ''.}
\thispagestyle{empty}
\restoregeometry

\addtocontents{toc}{\protect\thispagestyle{empty}}

\chapter{Vectors and Matrices}
\label{ch:vm}
\input{Chapters/ch1.tex}

\chapter{Functions of Several Variables}
\label{ch:fsv}
\input{Chapters/ch2.tex}

\chapter{Integration \texttt{(written; to be expanded and brushed up)}}
\label{ch:i}
\input{Chapters/ch3.tex}

\chapter{Differential Equations \texttt{(written; to be exp.~and brushed up)}}
\label{ch:de}
\input{Chapters/ch4.tex}

\printendnotes

\end{document}